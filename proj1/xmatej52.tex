\documentclass[a4paper, 11pt, twocolumn]{article}
\usepackage[left=2cm,text={17cm, 24cm},top=2.5cm]{geometry}
\usepackage[utf8]{inputenc}
\usepackage[czech]{babel}
\usepackage{times}

\newcommand{\czuv}[1]{\quotedblbase #1\textquotedblleft}
%opening
\title{
  Typografie a publikování\\
  1. projekt\\
}
  
\author{
  Jiří Matějka\\
  xmatej52@stud.fit.vutbr.cz
  \date{}\\
}

\begin{document}
  \maketitle
  \section{Hladká sazba}
	Hladká sazba je sazba z jednoho stupně, druhu a řezu písma sázená na stanovenou
	šířku odstavce. Skládá se z odstavců, které obvykle začínají zarážkou, ale mohou
	být sázeny i bez zarážky - rozhodující je celková grafická úprava. Odstavce
	jsou ukončeny východovou řádkou. Věty nesmějí začínat číslicí.
	
	
	Barevné zvýraznění, podtrhávání slov či různé velikosti písma vybraných slov
	se zde také nepoužívá. Hladká sazba je určena především pro delší texty,
	jako je například beletrie. Porušení konzistence sazby působí v textu
	rušivě a unavuje čtenářův zrak. 
  \section{Smíšená sazba}
	Smíšená sazba má o něco volnější pravidla než hladká sazba. Nejčastěji se
	klasická hladká sazba doplňuje o další řezy písma pro zvýraznění důležitých
	pojmů. Existuje \czuv{pravidlo}:
    \begin{center}
    \begin{quotation}
    \textsc{
      Čím více druhů, řezů, velikostí, barev písma a jiných efektů
      použijeme, tím profesionálněji bude dokument vypadat. čtenářův
      tím bude vždy nadšen!
    }
    \end{quotation}
    \end{center}
    Tímto pravidlem se \underline{nikdy} nesmíte řídit. Příliš časté zvýrazňování
    textových elementů a změny
    \LARGE{V}\Large{E}\large{L}\normalsize{IK}\small{O}\footnotesize{S}\scriptsize{T}\tiny{I}
    \normalsize{písma} \large{jsou} \Large{známkou} \textbf{\LARGE {amatérismu}}
    \normalsize autora a působí \emph{\textbf{velmi} rušivě}. Dobře navržený
    dokument nemá obsahovat více něž 4 řezy či druhy písma. \texttt{Dobře
    navržený dokument je decentní, ne chaotický\\}
    Důležitým znakem správně vysázeného dokumentu je konzistentní používání
    různých druhů zvýraznění. To například může znamenat, že \textbf{tučný řez}
    písma bude vyhrazena pouze pro klíčová slova, \emph{skloněný řez} pouze
    pro doposud neznámé pojmy a nebude se to míchat. Skloněný řez nepůsobí tak
    rušivě a používá se častěji. V \LaTeX u jej sázíme raději příkazem
    \verb \emph{text}  než \verb \textit{text} .
    
    
    Smíšená sazba se nejčastěji používá pro sazbu vědeckých článků a technických zpráv.
    U delších dokumentů vědeckého či technicekého charakteru je zvykem
    upozornit čtenáře na význam různých tupů zvýraznění v úvodní kapitole.  
    \section{České odlišnosti}
    Česká sazba se oproti okolnímu světu v některých aspektech mírně
    liší. Jednou z odlišností je sazba uvozovek. Uvozovky se v češtině
    používají převážně pro zobrazení přímé řeči. V menší míře se používají
    také pro zvýraznění přezdívek a ironie. V češtině se používá \czuv{typ
    uvozovek} namísto anglických "uvozovek".
    
    
    Ve smíšené sazbě se řez uvozovek řídí řezem prvního uvozeného slova.
    Pokud je uvozována celá věta, sází se koncová tečka před uvozovku.
    

    Druhou odlišností je pravidlo sázení konců řádků. V české sazbě by
    řádek neměl končit osamocenou jednopísmenou předložkou nebo spojkou
    (spojkou \czuv{a} končit může při sazbě do 25 liter). Abychom
    \LaTeX u zabránili v sázení do osamocených předložek, vkládáme
    mezi předložku a slovo nezlomitelnou mezeru znakem $\sim$ (vlnka, tilda).
    Pro automatické doplnění vlnek slouží volně šířitelný program
    \emph{vlna} od pana Olšáka\footnote{Viz ftp://math.feld.cvut.cz/pub/olsak/vlna/.}.    
\end{document}
