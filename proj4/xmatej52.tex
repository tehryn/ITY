%2/2 monografie (alespoň jedna musí být cizojazyčná),
%3/3 elektronické dokumenty (online),
%1/1 seriálovou publikaci (tištěný časopis či sborník z konference),
%2/2 články v seriálové publikaci (pokud možno zahraniční tištěné články),
%2/2 kvalifikační práce (bakalářské, diplomové nebo disertační).
\documentclass[11pt,a4paper,titlepage]{article}
\usepackage[left=2cm,text={17cm, 24cm},top=3cm]{geometry}
\usepackage{times}
\usepackage[czech]{babel}
\usepackage[utf8]{inputenc}
\bibliographystyle{czplain}
\usepackage{etoolbox}
\apptocmd{\thebibliography}{\raggedright}{}{}
%\usepackage{hyperref}
%\usepackage[hyphenbreaks]{breakurl}

\begin{document}
%%%%%%%%%%%%%%%%%%%%%%%FRONTPAGE%%%%%%%%%%%%%%%%%%%%%%%%%%%%%%%%%%%%
\thispagestyle{empty}
\begin{center}

\textsc{
\Huge{Vysoké učení technické v~Brně \\}
\huge{Fakulta informačních technologií\\}
}
\vspace{\stretch{0.382}}


\LARGE{Typografie a publikování -- 4. projekt \\}
\Huge{Bibliografické citace\\}
\vspace{\stretch{0.618}}

\end{center}

\Large{\today} \hfill Jiří Matějka

%%%%%%%%%%%%%%%%%%%%%%%%%%%%%%%%%%%%%%%%%%%%%%%%%%%%%%%%%%%%%%%%%%%%%%
\newpage
\setcounter{page}{1}
\section{\LaTeX}
\subsection{Co to je \LaTeX ?} 
Velice stručně můžeme říct, že
%https://books.google.cz/books?id=BTPgB6dzhG4C&printsec=frontcover&dq=Latex&hl=cs&sa=X&redir_esc=y#v=onepage&q=Latex&f=false
\LaTeX je komplexní sada příkazů používaných s~velice užitečným sázecím programem Tex pro přípravu nejrůznějších dokumentů.
\cite{book:guide-to-latex}
Ovšem toto je velice stručná definice a zdaleka nevystihuje všechny významy
ani užití tohoto mocného a velice často užívaného nástroje.
\LaTeX se užívá zejména pro psaní odborných článků, kvalifikačních prací, reportáží a v~mnoha dalších oblastech.
Oproti Tex je značně jednodušší, a proto patří mezi nejvhodnější nástroje ve~své
oblasti a to nejen díky své jednoduchosti, ale i proto, že je, stejně jako Tex,
zdarma k~užití.
% 4 vety

\subsection{Proč používat \LaTeX ?}
%http://www.latextemplates.com/why-use-latex
Jednoduše můžeme říct, že dokumenty vytvořené pomocí \LaTeX u vypadají prostě lépe.
Navíc \LaTeX~je zdarma a multiplatformní, tedy i lehce přenositelný. Další výhodou je, že Tex
soubor je pouhý zdrojový kód, tedy jednoduchý text a může být upraven v~každém
textovém editoru.\cite{web:why-use-latex}
%7 vet

\subsection{Libre office nebo \LaTeX ?}
%http://mappingignorance.org/2015/04/06/word-or-latex-typesetting-which-one-is-more-productive-finally-scientifically-assessed/
Výběr vhodného "nástroje" záleží na velikosti dokumentu. Při psaní krátkých textů
bude vhodnější sada Microsoft office, Libre office a další podobné nástroje. Ale jak délka
dokumentu roste, budete muset vynaložit větší a větší úsilí na formátování textu
a celkového vzhledu dokumentu, kdeždo při použití \LaTeX u si nadefinujete formát
stránky, písma, vlastní příkazy a ty můžete použít v~celém dokumentu.\cite{web:LibreOfficevsLatex}
% 9 vet

\subsection{Nevýhody při používání \LaTeX u}
%http://merlin.fit.vutbr.cz/wiki/index.php/LaTeX#.22Nev.C3.BDhody.22_syst.C3.A9mu_LaTeX
Není úplně jednoduché se \LaTeX~naučit a osvojit si ho zabere určitý čas.
Navíc při samotném zpracování dokumentu není do jeho překladu vidět výsledek práce.
A~v~neposlední řadě je potřeba mít nainstalované určité nástroje, hlavně při použití
české diakritiky.\cite{web:nevyhody-latex}
%12 vet

\newpage
\section{Operační systémy}
\subsection{Mikrojádro}
%https://www.vutbr.cz/studium/zaverecne-prace?action=detail&zp_id=15049&fid=20042&rok=&typ=&jazyk=cs&text=linux&hl_klic_slova=1&hl_abstrakt=0&hl_nazev=1&hl_autor=0&str=1 strana 8 dole
Mikrojádro je takové jádro operačního systému, které obsahuje pouze nezbytně nutné
služby operačního systému. Nezbytně nutné se rozumí takové, které jsou potřeba pro
"komunikaci" operačního systému s~hardwarem (správa procesů, správa paměti 
a~funkce pro synchronizaci a komunikaci).\cite{bc:mikrojadra}
%14 vet

%https://books.google.cz/books?id=wOJjBAAAQBAJ&pg=PA37&lpg=PA37&dq=issue+magazine+kernel+process+management&source=bl&ots=Kfz778RDp6&sig=GqHCecZI9J8FrKjO0mDnO43J9pk&hl=cs&sa=X&ved=0ahUKEwjszanS-YPMAhXjFZoKHaB1B0UQ6AEIOTAE#v=onepage&q=issue%20magazine%20kernel%20process%20management&f=false strana 37
\subsection{Správa procesů a jejich přístupu k~CPU}
V~dnešní době mají počítače jeden a více procesorů, avšak každý procesor dovede
ve~skutečnosti pracovat pouze s~jedním procesem, přestože jich zvládá obsluhovat
více. Ve~skutečnosti tedy obsluhuje v~jednu chvíli jeden proces po určitou dobu
a poté obsluhuje jiný a neustále se přepíná mezi přidělenými procesy. Zatímco
se procesor věnuje jednomu procesu, ostatní procesy spí.\cite{article:memory-management}
%17 vet

%https://www.vutbr.cz/www_base/zav_prace_soubor_verejne.php?file_id=115831 strana 12
\subsection{Správa paměti}
Paměť patří mezi nejdůležitější prvky počítače a má veliký vliv na výkonnost
celého stroje, proto správa paměti musí být co nejefektivněji a~zároveň co nejbezpečněji
implementována. Musí být efektivně rozdělována mezi uživatele, procesy
a pro samotné jádro operačního systému a musí být prováděna i dealokace paměti
v~každou chvíli, kdy je to vhodné a bezpečné.\cite{bc:linux-pamet}
%19 vet

\subsubsection{Problémy spojené se správou pamětí}
V~dnešní době multiprocesingu, kdy je spuštěno poměrně velké množství procesů,
je správa paměti poměrně problematická, zejména ve~sdílených částech paměti, kde
si můžou poměrně jednoduše procesy navzájem přepisovat paměť a~tak značně měnit
své chování.
%20

\newpage
\bibliography{xmatej52}
\end{document}
