\documentclass[a4paper, 11pt, twocolumn]{article}
\usepackage[left=1.5cm,text={18cm, 25cm},top=2.5cm]{geometry}
\usepackage[utf8]{inputenc}
\usepackage[czech]{babel}
\usepackage{setspace}
\usepackage{times}
\usepackage{amsthm}
\usepackage{amsmath}
\usepackage{amssymb}
\usepackage{mathtools}
\newcommand{\czuv}[1]{\quotedblbase #1\textquotedblleft}
  
% \author{
%   Jiří Matějka\\
%   xmatej52@stud.fit.vutbr.cz
%   \date{}\\
% }
\theoremstyle{plain}
 \newtheorem{thm}{Theorem}[section]

\theoremstyle{definition}
\newtheorem{defn}[thm]{Definice}
\newtheorem{alg}[thm]{Algoritmus}

\theoremstyle{plain}
\newtheorem{lemma}{Lemma}[section]

\newtheorem{sen}{Věta}

\begin{document}
\onecolumn
{\setstretch{0.5}
\begin{center}

\Huge
\textsc{
Fakulta informačních technologií \\
Vysoké učení technické v Brně}\\
\vspace{\stretch{0.382}}


\LARGE
Typografie a publikování -- 2. projekt \\
Sazba dokumentů a matematických výrazů\\
\vspace{\stretch{0.618}}

\end{center}

\Large{2016} \hfill Jiří Matějka
}% AND singlespacing ONECOLUMN
\thispagestyle{empty}


\newpage
\setcounter{page}{1}
\twocolumn


\section*{Úvod}
V této úloze si vyzkoušíme sazbu titulní strany, matematických vzorců, prostředí
a dalších textových struktur obvyklých pro technicky zaměřené texty, například
rovnice \eqref{eq:1} nebo definice \ref{sec:df} na straně \pageref{sec:df}


Na titulní straně je využito sázení nadpisu podle optického středu s využitím zlatého řezu.
Tento postup byl probírán na přednášce.

\section{Matematický text}
Nejprve se podíváme na sázení matematických symbolů a výrazů v plynulém textu.
Pro množinu $V$ označuje card$(V)$ kardinalitu $V$. Pro množinu $V$ reprezentuje
$V^{*}$ volný monoid generovaný množinou $V$ s operací konkatenace. Prvek
identity ve volném monoidu $V^{*}$ značíme symbolem $\varepsilon$ Nechť
$V^{+} = V^{*} - \{ \varepsilon \}$. Algebraicky je tedy $V^{+}$ volná pologrupa
generovaná množinou $V$ s operací konkatenace. Konečnou neprázdnou množinu $V$
nazvěme \emph{abeceda}. Pro $w \in V^{*}$  označuje $|w|$ délku řetězce $w$. Pro
$W \subseteq V$ označuje occur$(w, W)$ počet výskytů symbolů z $W$ v řetězci $w$
a sym$(w, i)$ určuje $i$-tý symbol řetězce $w$; například sym$(abcd, 3) = c$


Nyní zkusíme sazbu definic a vět s využitím balíku \texttt{amsthm}.


\begin{defn}
\emph{Bezkontextová gramatika} je čtveřice $G = (V, T, P, S)$, kde $V$ je
totální abeceda, $T \subseteq V$ je abeceda terminálů, $S \in (V - T)$ je
startující symbol a $P$ je konečná množina \emph{pravidel} tvaru
$q\colon A \rightarrow \alpha$, kde $A \in (V - T), \alpha \in V^{*}$ a $q$ je
návěští tohoto pravidla. Nechť $N = V - T$ značí abecedu neterminálů. Pokud
$q\colon A \rightarrow \alpha \in P$ , $\gamma, \delta \in V^{*}$, $G$ provádí
derivační krok z $\gamma A \delta$ do $\gamma \alpha \delta$ podle pravidla
$q\colon A \rightarrow \alpha$, symbolicky píšeme 
$\gamma A \delta \Rightarrow \gamma \alpha \delta [ q\colon A \rightarrow \alpha] $
nebo zjednodušeně $\gamma A \delta \Rightarrow \gamma \alpha \delta $ .
Standardním způsobem definujeme $\Rightarrow^{m}$, kde $m \geq 0$ .
Dále definujeme tranzitivní uzávěr $\Rightarrow^{+}$ a tranzitivně-reflexivní
uzávěr $\Rightarrow^{*}$
\end{defn}

Algoritmus můžeme uvádět podobně jako definice textově, nebo využít pseudokódu
vysázeného ve vhodném prostředí (například \texttt{algorithm2e}).


\begin{alg}
\emph {
Algoritmus pro ověření bezkontextovosti gramatiky. Mějme gramatiku
G = (N, T, P, S).
\begin{enumerate}
\item \label{item}Pro každé pravidlo $p \in P$ proveď test, zda $p$ na levé
straně obsahuje právě jeden symbol z $N$ .
\item Pokud všechna pravidla splňují podmínku z kroku~\ref{item}, tak je
gramatika $G$ bezkontextová.
\end{enumerate}
}
\end{alg}

\begin{defn}
\emph{Jazyk} definovaný gramatikou $G$ definujeme jako $L(G) = \{ w \in T^{*} | S \Rightarrow^{*} \} $.
\end{defn}

\subsection{Podsekce obsahující větu} \label{sec:df} 

\begin{defn}
Nechť $L$ je libovolný jazyk. $L$ je \emph{bezkontextový jazyk}, když a jen
když $L = L(G)$, kde $G$ je libovolná bezkontextová gramatika.
\end{defn}

\begin{defn}
Množinu $\mathcal{L}_{CF} = \{ L|L $ je bezkontextový jazyk$\}$ nazýváme % \text{} a \textrm{} mi zde z neznamych duvodu nefunguje
\emph{třídou bezkontextových jazyků.}
\end{defn}

\begin{sen}
\label{sentence}Nechť $L_{abc} = \{a^{n}b^{n}c^{n}|n \geq 0\}$. Platí, že $L_{abc} \notin \mathcal{L}_{CF}$.
\end{sen}

\begin{proof}
Důkaz se provede pomocí Pumping lemma pro bezkontextové jazyky, kdy ukážeme,
že není možné, aby platilo, což bude implikovat pravdivost věty \ref{sentence} .
\end{proof}

\section{Rovnice a odkazy}
Složitější matematické formulace sázíme mimo plynulý text. Lze umístit
několik výrazů na jeden řádek, ale pakje třeba tyto vhodně oddělit, například
příkazem \verb \quad .

$$
\sqrt[x^2]{y_{0}^{3}}
\quad
\mathbb{N}= \left\{0,1,2,... \right \}
\quad
x^{{y}^y}  \neq x^{yy}
\quad
z_{i_{j}} \not\equiv z_{ij}
$$

 V rovnici \eqref{eq:1} jsou využity tři typy závorek s různou explicitně definovanou velikostí.
\begin{align}
\bigg\{ \Big[ \big( a+b \big) *c  \Big] ^{d} +1\ \bigg\} \quad &= \quad x   \label{eq:1} \\  
\lim_{x \to \infty} \frac{\sin^{2}x + \cos^{2}x}{4} \quad &= \quad y \nonumber
\end{align}

V této větě vidíme, jak vypadá implicitní vysázení limity $\lim_{x\to\infty} f(n)$
v normálním odstavci textu. Podobně je to i s dalšími symboly jako
$\sum\nolimits_{1}^{n} $ či $\bigcup_{A \in B}$ . V případě vzorce 
$\lim\limits_{x\to0}\frac{\sin x}{x} = 1$ jsme si vynutili méně úspornou
sazbu příkazem \verb \limits .

\begin{align}
\intop_{a}^{b} f(x)\:\mathrm{d}x \quad &= \quad - \int_{a}^{b} f(x)\:\mathrm{d}x \\
\left( \sqrt[5]{x^{4}} \right) = \left(x^{\frac{4}{5}}  \right)' \quad &= \quad \frac{4}{5}x^{-\frac{1}{5}} = \frac{4}{5\sqrt[5]{x}} \\
\overline{\overline{A \vee B}} \quad &= \quad \overline{\overline{A} \wedge \overline{B}}
\end{align}
 
\section{Matice}
Pro sázení matic se velmi často používá prostředí \texttt{array} a závorky
(\verb \left , \verb \right ).

\newpage

$$
\begin{pmatrix}		
a+b & b-a\\ 
\widehat{\xi + \omega}  & \hat{\pi }\\
\vec{a} & \overleftrightarrow{AC}  \\
0 & \beta 
\end{pmatrix}
$$

$$
\mathbf{A} =
\begin{Vmatrix}
a_{11} & a_{12} & \cdots & a_{1n} \\
a_{21} & a_{22} & \cdots & a_{2n} \\ 
\vdots & \vdots & \ddots & \vdots \\ 
a_{m1} & a_{m2} & \cdots & a_{mn}
\end{Vmatrix}
$$

$$
\begin{vmatrix}
t & u\\ 
v & w
\end{vmatrix}
= tw - uv
$$

Prostředí \texttt{array} lze úspěšně využít i jinde.

$$
\begin{pmatrix}
n\\ 
k
\end{pmatrix}
= \Bigg\{\
\begin{matrix*}[l]
\frac{n!}{k!(n-k)!} & \mathrm{pro}\: 0\leq k \leq n \\
0 & \mathrm{pro}\: k<0\: \mathrm{nebo}\:k>n
\end{matrix*}
$$

\section{Závěrem}
V případě, že budete potřebovat vyjádřit matematickou konstrukci nebo
symbol a nebude se Vám dařit jej nalézt v samotném \LaTeX u,
doporučuji prostudovat možnosti balíku maker \AmS-\LaTeX.

\end{document}